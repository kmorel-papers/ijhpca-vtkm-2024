\section{Integration into Visualization Tools}

Throughout the lifespan of ECP, the VTK-m team operated in heavy collaboration with other ECP software technology teams.
The scope of the VTK-m portion of the project was to provide the fundamental technology to run scientific visualization algorithms on the GPU processors of the Exascale machines.
Other ECP teams, most notably the ALPINE project, developed tools that would leverage VTK-m while directly addressing application needs.
This arrangement avoided the redundant work of multiple teams developing their own visualization solutions and prevented users from having to use yet another software interface.
In this section we discuss the major visualization tools we integrated VTK-m with.

\ken{
  Each subsection should be roughly 1/2 page plus have an image demonstrating the tool with VTK-m that is about 1/3 page.
  (3 + 1/3 page total.)
  The subsection should start with a description of the tool.
  (Exception: the last subsection starts with a description of the lengthy process from committing code in VTK-m to it being available in a tool.)
  The following paragraphs describe how VTK-m is integrated at a high level.
  Avoid details like classnames.
}

\subsection{ParaView}

\assign{Sujin}

\subsection{VisIt}

\assign{Eric}

\subsection{Ascent}

\assign{Nicole}

\subsection{Alternate Delivery Mechanisms}

\assign{Tushar}

\ken{
  Talk about how you can deliver new functionality to these tools outside of the pipeline of implement in VTK-m $\rightarrow$ VTK $\rightarrow$ Tool.
  Use uncertainty plugin as an example of doing this.
}
