\section{Integration into Visualization Tools}

Throughout the lifespan of ECP, the \vtkm team operated in heavy collaboration with other ECP software technology teams.
The scope of the \vtkm portion of the project was to provide the fundamental technology to run scientific visualization algorithms on the GPU processors of the Exascale machines.
Other ECP teams, most notably the ALPINE project, developed tools that would leverage \vtkm while directly addressing application needs.
This arrangement avoided the redundant work of multiple teams developing their own visualization solutions and prevented users from having to use yet another software interface.
In this section we discuss the major visualization tools we integrated \vtkm with.

%% \ken{
%%   Each subsection should be roughly 1/2 page plus have an image demonstrating the tool with \vtkm that is about 1/3 page.
%%   (3 + 1/3 page total.)
%%   The subsection should start with a description of the tool.
%%   (Exception: the last subsection starts with a description of the lengthy process from committing code in \vtkm to it being available in a tool.)
%%   The following paragraphs describe how \vtkm is integrated at a high level.
%%   Avoid details like classnames.
%% }

\subsection{ParaView}

\begin{figure}[htb]
  \includegraphics[width=\linewidth]{figures/paraview-crusher.png}
  \includegraphics[width=\linewidth]{figures/threshold-vtkm-gpu-usage-crusher-small.png}
  \caption{
    ParaView, with integrated \vtkm accelerated filters, running on Crusher, an early access test bed for the Frontier system.
    \vtkm is using the Kokkos device adapter.
    The output of the \texttt{rocm-smi} command is being used to verify and monitor GPU usage by the filters.
    %\ken{I think it would be better to move this figure (or something like it) to the ParaView section.}
  }
  \label{fig:paraview-crusher}
\end{figure}

%\assign{Sujin}
The \vtkm library provides high performance implementations of several visualization algorithms for highly parallel processors.
However, features such as file I/O, rendering, and pipeline management, which are essential parts of a full featured visualization toolkit, are beyond the scope of \vtkm.
On the other hand, ParaView is a mature visualization software that has robust implementation of these features.
Therefore, we wanted to integrate \vtkm into ParaView, such that ParaView can use \vtkm filters to accelerate its operations when a \vtkm implementation and highly parallel hardware is available.

We wanted to make the use of \vtkm accelerated filters as effortless as possible.
Therefore, we have chosen to integrate \vtkm using VTK and ParaView's factory-instantiation feature.
Since ParaView is implemented on top of VTK and internally relies on VTK filters, both VTK and ParaView will be mentioned interchangeably in this section.
Filters in ParaView are instantiated via a factory method.
There can be multiple implementations available for a filter, and the factory method chooses an appropriate implementation at run-time based on some criteria.
With this method we can override the default ParaView filters with \vtkm based filters.
Currently, \vtkm accelerated filters that override traditional CPU implementations are available for some commonly used filters such as Contour, Threshold, and Gradient.
%Adding overrides for more filters is a straightforward process as discussed in the following paragraphs.
Further \vtkm filters can be added by providing additional overrides.

To override a ParaView filter, we first need to implement a \vtkm wrapper filter in VTK that provides the interface of the base VTK/ParaView filter and uses \vtkm filters and routines for its operation. The following steps give a high-level overview of how a \vtkm wrapper filter is implemented in VTK/ParaView.
\begin{enumerate}
    \item Check the filter parameters and only proceed with \vtkm processing for configurations supported by the \vtkm filter implementation.
    \item Convert the input VTK datasets to \vtkm datasets. 
    \item Execute the \vtkm filter on the data.
    \item Convert the output of the \vtkm filter back to VTK datasets
    \item If at any point during the above steps there is an error, fall back to the default VTK implementation. Errors in \vtkm are typically signaled via C++ exceptions.
\end{enumerate}

For the dataset conversion from VTK to \vtkm and back, we have implemented several helper routines. These are zero-copy operations for most cases as only the ownership of the pointers to the underlying resources are transferred wherever possible. Even copy from host to device and device to host are minimized with the use of a \vtkm dataset wrapper in VTK called \texttt{vtkmDataSet}, which implements the interface of \texttt{vtkDataSet} and only copies the data when required. Another commonly used ParaView functionality of computing the range of the various fields of a dataset has also been accelerated using \vtkm to speed up the computation and to avoid memory transfer from device to host.

Figure~\ref{fig:paraview-crusher} shows an example of ParaView running on Crusher, which is an early access test bed for the Frontier system, with \vtkm accelerated filters.
As described in section \nameref{sec:adopting-kokkos}, \vtkm is using the Kokkos device adapter backend on this hardware.
The bottom of the image shows the output of the \texttt{rocm-smi} command, which is used to verify and monitor the GPU usage by the filters.

\begin{figure}[htb]
  \includegraphics[width=\linewidth]{figures/pv-override-settings.png}
  \caption{
    Left: The ParaView setting \texttt{Use Accelerated Filters} can be used to turn on or off accelerated filter overrides at run-time.
    This setting is shown irrespective of whether ParaView was built with the override support or not.
    Right: The availability of the accelerated filter overrides on clients and servers can be found in the \texttt{About ParaView} dialog box.
  }
  \label{fig:paraview_settings}
\end{figure}

%\ken{This last paragraph might be more detail than needed.}
% The wrapper filter needs to be registered with the factory for the factory instantiation to work. This is done at the build configuration using the CMake function:
% \texttt{\_vtkm\_add\_override("vtkBaseFilter" "vtkmAcceleratedFilter")} Then, the accelerated filters need to be enabled using the CMake option \texttt{VTK\_ENABLE\_VTKM\_OVERRIDES}. They can also be turned on or off during run time through ParaView settings as shown in Figure~\ref{fig:paraview_settings}.
%\sujin{Ken, please review if the changes are satisfactory}
Accelerated \vtkm filter overrids are now available in the recent releases of ParaView if the feature was enabled during building.
If enabled, this feature can also be turned on or off at run-time using ParaView settings as described in Figure~\ref{fig:paraview_settings}.

\subsection{VisIt}

%\assign{Eric}
VisIt is a scientific visualization and analysis tool that operates on mesh-based field data.
Its functionality is grouped into four major categories: Plots, Operators, Expressions and Queries.
All four of these capabilities are built on a filter infrastructure that operates on mesh-based fields.
Plots are somewhat special in that they consist of a rendering capability that may include some built-in filter operations.
The \vtkm integration to date has consisted of modifying the filter infrastructure to use \vtkm filters where there exists comparable \vtkm functionality.
%There has not yet been an effort to utilize \vtkm's rendering capabilities.


Previously, VisIt's filters used VTK filters and VTK data sets.
The filters were enhanced to support using both VTK and \vtkm.
When \vtkm is enabled in VisIt and the filter supports \vtkm, the filter will use \vtkm.
The internal data set representation was also modified to support providing either a VTK data set or a \vtkm data set.
When the filter wants to use VTK, it will request the data as a VTK data set and convert the data set to a VTK data set if it is stored as a \vtkm data set.
Conversely, when the filter wants to use \vtkm, it will request the data as a \vtkm data set and convert it if necessary.
When doing the conversions, it will use zero-copy conversions wherever possible.

\begin{figure}[htb]
  \includegraphics[width=\linewidth]{figures/visit_warpx_frontier.png}
  \caption{Visualization from a 70 billion cell WarpX Gordon Bell simulation~\cite{FedeliHuebl2022} visualized with 2048 GCDs on Frontier using VisIt.}
  \label{fig:visit_warpx_frontier}
\end{figure}

Figure~\ref{fig:visit_warpx_frontier} is an image generated by VisIt running on Frontier using 2048 GCDs on 256 nodes.
The surfaces were generated using the \vtkm Contour filter and were rendered in parallel using Mesa 3D.
\vtkm is using the Kokkos backend for AMD GPUs.

\subsection{Ascent}

%\assign{Nicole}
Ascent is a lightweight, in situ visualization and analysis library designed for multi-physics HPC simulations. As an in situ library, as opposed to a post-hoc visualization tool, Ascent shares execution resources with the simulation and can process the data as it is generated, reducing I/O costs, though it has to pause the simulation to do so. To minimize the encumbrance on the simulation and execution resources, Ascent's lightweight design ensures a small memory requirement. It is written using efficient distributed-memory and many-core libraries to guarantee performance and scalability on current and next-generation HPC platforms. Ascent has three main use cases: making pictures, transforming data, and capturing data. Ascent aims to be easy-to-use with only five API calls supported in C, C++, Python, and Fortran, while also providing an infrastructure to integrate custom analysis.
The data interface between simulation code and Ascent is managed through the Conduit API \cite{Harrison2022}, which provides a simplified interface for passing data and describing structure.

\vtkm, while optional, is a main dependency for Ascent, as it is currently the only option for rendering low-order mesh data offered in Ascent, and provides filters for transforming and/or analyzing the simulation data e.g.
Slice, Histogram, Contour.
Ascent also takes advantage of \vtkm's zero-copy capabilities as well as its ability to pass device-pointers, allowing the simulation data to remain on the device and be passed directly to Ascent, and then on to \vtkm, without having to be transferred to host memory.
\vtkm has been integrated into Ascent via a (previously external) library called VTK-h (the ``h'' meaning ``hybrid''), that combines \vtkm's shared-memory performant filters with MPI's efficient distributed-memory coordination. 

\begin{figure}[b]
    \centering
    \includegraphics[width=\linewidth]{figures/ez_007050.png}
    \caption{WarpX in situ visualization of a laser-wakefield accelerator on 552 nodes containing 4,416 GCDs on Frontier using Ascent and \vtkm. This shows an early timestep of the simulation at a high resolution.}
    \label{fig:warpx_highres}
\end{figure}

\begin{figure}[b]
    \centering
    \includegraphics[width=\linewidth]{figures/ey_009300.png}
    \caption{WarpX in situ visualization of a laser-wakefield accelerator on 69 nodes containing 552 GCDs on Frontier using Ascent and \vtkm. This shows a later timestep of the simulation at low resolution.}
    \label{fig:warpx_lowres}
\end{figure}

Figures~\ref{fig:warpx_highres} and~\ref{fig:warpx_lowres} show in situ renderings of the WarpX simulation (see the section \nameref{sec:warpx} later in this paper) generated by Ascent executing on OLCF's Exascale supercomputer Frontier.
The simulation was executed in two resolutions: 578.8 million cells across 552 GCDs on 69 nodes and 4.63 billion cells across 4,416 GCDs on 552 nodes.
Ascent used \vtkm filters to upscale the data along multiple axes, generate isosurfaces, and clip several fields, before using \vtkm's raytracer and volume renderer to generate the final images.
To guarantee performance, \vtkm uses the Kokkos backend for AMD GPUs. 

%\begin{figure}[htb]
%  \includegraphics[width=\linewidth]{figures/warpx_stages_lwfa.png}
%  \caption{WarpX in situ visualization of a laser-wakefield accelerator on 552 nodes containing 4,416 GPUs on Frontier using Ascent and \vtkm.
%  The inset shows an early simulation time step at high resolution.}
%  \label{fig:warpx_lwfa}
%\end{figure}

\subsection{Alternate Delivery Mechanisms}

%\assign{Tushar}

Integrating a new feature implemented with \vtkm filters into visualization software such as ParaView and VisIt can be a lengthy process.
For example, making a \vtkm filter available in ParaView requires going through multiple steps, including implementing a VTK filter that wraps the \vtkm filter, completing the arduous process of contributing the change to the VTK project, and repeating similar steps in ParaView.
Such time-consuming software integration can hinder the availability of \vtkm filters inside visualization tools, thus reducing the opportunities for \vtkm filters to enhance the pace of scientific discovery.

Our ultimate goal is to make \vtkm filters practical for real use and to put tools in the hands of end users in a timely manner.
An alternate approach to a full integration through the visualization software stack is to provide this functionality instead through a plugin, which is supported by tools like ParaView and VisIt.
For the plugin approach, the \vtkm filter still needs to be wrapped inside a VTK filter, but the time needed for the software integration and testing in VTK and ParaView can be bypassed.

Figure~\ref{fig:uncertainty-plugin} illustrates the \vtkm isosurface uncertainty filter~\cite{Wang2023, Athawale21} made available in ParaView using the plugin method. The isosurface uncertainty filter is one of the major successes of the \vtkm library, as it is the first production-level uncertainty visualization filter deployed for efficient large-data analysis. Using the plugin method depicted in Figure~\ref{fig:uncertainty-plugin}, \vtkm filters can be easily coupled with existing filters in ParaView for faster and better data understanding.       

\begin{figure}[htb]
  \includegraphics[width=\linewidth]{figures/isosurfaceUncertaintyPlugin.png}
  \caption{Integration of the \vtkm isosurface uncertainty filter into ParaView using the plugin approach for visualization of large-scale supernova simulations~\cite{Sandoval2021}.}
  \label{fig:uncertainty-plugin}
\end{figure}



%% \ken{
%%   Talk about how you can deliver new functionality to these tools outside of the pipeline of implement in \vtkm $\rightarrow$ VTK $\rightarrow$ Tool.
%%   Use uncertainty plugin as an example of doing this.
%% }
