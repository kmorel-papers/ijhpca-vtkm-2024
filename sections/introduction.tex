\section{Introduction}

The Exascale Computing Project (ECP) brought high-performance computing to the next milestone of capacity computing.
Although the goal was nominally to stand up computers capable of a billion billion (1018) floating point operations per second, the advancement required a revolution on hardware and software to achieve this goal.
Pivotal to this advancement is the adoption of GPU processors from a variety of vendors as the primary computing engine.
These GPUs provide an unmatched computational throughput for the power they consume at the cost of greater code complexity.

As scientific simulations consume more computation, they produce more data.
To understand data and promote discovery, these data must be analyzed and visualized, a high-performance task onto itself.
VTK-m is the software library that makes this possible by implementing classic scientific visualization algorithms redesigned for heavily threaded environments like GPUs.
VTK-m also provides a framework that simplifies the implementation of new algorithms and provides a porting layer to work across multiple processor types.
At the inception of ECP, VTK-m was the byproduct of a research project.
ECP provided the investment to promote VTK-m to production software that now serves as the underlying visualization implementation across all ECP.
VTK-m is currently used by ParaView, VisIt, and Ascent to execute visualization algorithms on Exascale platforms and others using GPU processing.

This paper provides an overview of the main challenges encountered to make visualization available at the Exascale.
This includes porting challenges, performance testing and improvement, integration with other ECP software technologies, and the support of ECP applications.
